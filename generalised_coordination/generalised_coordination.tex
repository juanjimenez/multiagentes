\documentclass{beamer}
\usetheme{default} %Montpellier}
\usepackage{amsmath,amsthm}
\usepackage{tikz}
\usetikzlibrary{shapes.geometric}
\usetikzlibrary{shapes.multipart}
\usetikzlibrary{decorations.pathreplacing}
\usepackage[utf8]{inputenc}
\usepackage[english]{babel}
\usepackage{amstext}
\usepackage{amsfonts}
\usepackage{amssymb}
\title{Generalised Coordination of Multi-robot System}
\author{Juan Jiménez}
%\setbeamertemplate{navigation symbols}{} 
\logo{\includegraphics[width=.08\textwidth]{irobocity2030.jpg}}
\institute{Universidad Complutense de Madrid \\Dto. de  Arquitectura de Computadores y Automática.} 
\date{\today}
\setbeamertemplate{footline}[frame number]
\AtBeginSubsection[]
{
	\begin{frame}<beamer>
		\frametitle{índice}
		\tableofcontents[currentsection,currentsubsection]
	\end{frame}
} 
\begin{document}
\begin{frame}[plain]
    \maketitle
\end{frame}
\begin{frame}{Preliminary notions}
	We want to achieve a method to coordinate the evolution of a group of Agents in a common state space $\mathbb{R}^d$
	
\end{frame}
\begin{frame}{The Generalised Coordination Problem}

\begin{equation*}
	\lim_{t\to \infty} \text{dist}\left(x_{\mathcal{N}}(t),\mathcal{D}\right)=0,
\end{equation*}

where $\mathcal{D} \subset (\mathbb{R}^d)^n$ is a desired configuration set	
\end{frame}

\begin{frame}{Local and Global Coordinate Frames}
	
\begin{block}{We define two coordinate frames:}
	\begin{itemize}
		\item A Global (fixed to Earth) coordinate frame, $\Sigma$, Common to every agent
		\item A Local (fixed to body) coordinate frame $\Sigma_i$ for each agent $i$
	\end{itemize}
\end{block}	

A global coordinate $p(t) in \mathbb{R}^d$ and a local coordinate $p^{[i]}(t) \in \mathbb{R}^d$ are transformed into each other as,
\begin{equation*}
	p((t) = M_i(t)p^{[i]}(t) + b_i(t)
\end{equation*}
		Where $(M_i(t),b_i(t)) \in \mathcal{M}\ltimes \mathcal{B}$. 
		 
		$\mathcal{M}\ltimes \mathcal{B}$ is a frame transformation set, typically a subgroup of, $\text{scaled}(O(d))\ltimes \mathbb{R}^d$ 
\end{frame}

\begin{frame}{The problem}
\begin{block}{The Kinematic model}	
Given a transformation set $\mathcal{M}\ltimes \mathcal{B}$ and the coordinate transformation $p((t) = M_i(t)p^{[i]}(t) + b_i(t): (M_i(t),b_i(t)) \in \mathcal{M}\ltimes \mathcal{B}$ .
We define the kinematic model
\begin{equation*}
	\dot x_i(t) = M_i(t)u_i(t)
\end{equation*}  
\end{block}	
\end{frame}

\begin{frame}{The problem}
	\begin{block}{Local Coordinates}	
		The relative position $x^{[j]}_i$ of neighbor $j \in \mathcal{N}_i$ wrt agent $i$ is given as,
		\begin{equation*}
			x^{[j]}_i(t) = M_i^{-1}(t)(x_j(t)-b_i(t)) = (M_i(t),b_i(t))^{-1}\bullet x_j(t)
		\end{equation*}  
	\end{block}		
\end{frame}

\begin{frame}{The problem}
	\begin{block}{The admissible controller}	
		for a graph $G=(\mathcal{N},\mathcal{E})$ that sets the neighborhood of de agents 
		\begin{equation*}
			u_i(t) = c_i\left(x^{[i]}_{\mathcal{N}_i}(t)\right)
		\end{equation*}
		Where function $c_i:\left(\mathbb{R}^d\right)^{\vert \mathcal{N}_i \vert} \to \mathbb{R}^d$ and $\mathcal{N}_i \subset \mathcal{N}$ is the neighbor set of agent $i$.
	\end{block}
	
		$c_i$ is called a \emph{distributed controller with relative measurements}
			
\end{frame}

\begin{frame}{The problem}
	To achieve the generalised coordination with respect to a set $\mathcal{D} \subset \left(\mathbb{R}^d\right)^n$ we need to design a controller that asymptotically stabilises $\mathcal{D}$:
	\begin{block}{Gradient-flow approach}	
		\begin{equation*}
			x_i(t) = -k_i\frac{\partial v}{\partial x_i}\left(x_{\mathcal{N}}(t)\right)
		\end{equation*}
	\end{block}
	\begin{itemize}
		\item With a non-negative function $v:\left(\mathbb{R}^d\right)^n \to \mathbb{R}_+$ and a positive constant $k_i > 0$
		\item The objective function $v\left(x_{\mathcal{N}}\right)$ should be an indicator of $\mathcal{D}$ that is, $v\left(x_{\mathcal{N}}\right) \in \mathcal{V}_{ind}(\mathcal{D})$
	\end{itemize}
\end{frame}

\begin{frame}{The problem}
	
	\begin{block}{The controller}	
		\begin{equation*}
			c_i\left(x^{[i]}_{\mathcal{N}_i}(t)\right) = -k_iM_i^{-1}\frac{\partial v}{\partial x_i}\left(x_{\mathcal{N}}(t)\right)
		\end{equation*}
	\end{block}
	\begin{alertblock}{Recall:}
		$x_j^{[i]} = (M_i,b_i)^{-1}\bullet x_j $ represents the relative position of neighbor $j\in \mathcal{N}_i$ for $(M_i,b_i) \in \mathcal{M}\ltimes \mathcal{B}$ 
	\end{alertblock}
\end{frame}
\begin{frame}{The problem}
	
	\begin{block}{The objective function for the controller should belong to $\mathcal{V}_{rel}(\mathcal{M}\ltimes\mathcal{B})$}	
		\begin{multline*}
			\mathcal{V}_{rel}(\mathcal{M}\ltimes\mathcal{B}) =\{ v(x_{\mathcal{N}}) \in \mathcal{V}_{c1}:\forall i \in \mathcal{N}, \exists \bar{c}_i: (\mathbb{R}^d)^n \to \mathbb{R}^d \\ s.t.  ((M_i,b_i)^{-1}\frac{\partial v}{\partial x_i} (x_{\mathcal{N}})=\bar{c}_i\bullet x_{\mathcal{N}}) \}
		\end{multline*}
	\end{block}
	\begin{alertblock}{Recall:}
		$x_j^{[i]} = (M_i,b_i)^{-1}\bullet x_j $ represents the relative position of neighbor $j\in \mathcal{N}_i$ for $(M_i,b_i) \in \mathcal{M}\ltimes \mathcal{B}$ 
	\end{alertblock}
\end{frame}
\end{document}
