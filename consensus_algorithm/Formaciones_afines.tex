\documentclass[10pt,a4paper]{article}
\usepackage[utf8]{inputenc}
\usepackage[spanish]{babel}
\usepackage{amsmath}
\usepackage{amsfonts}
\usepackage{amssymb}
\usepackage{amsthm}
\usepackage{graphicx}
\newtheorem{lem}{lema}
\title{Notas en algoritmos de consenso y Formaciones Afines para multiagentes}
\begin{document}
\section{Introducción}
(Ideas tomadas de \cite{zhao2018}).
El control de formaciones multiagentes se compone de dos tareas distintas. La primera es el control de la forma que toma la formación. Consiste en conducir a un grupo de agentes móviles a formar un patrón geométrico deseado a partir de una configuración inicial arbitraria. La segunda segunda es el control de la maniobra de la formación. Consiste en conducir a los agente móviles a que se maniobren como un todo de modo que el centroide, la orientación, la escala y otras parámetros geométricos puedan ser controlados de forma continua. El control de maniobras es importante para conseguir que una formación de agentes puedan realizar tareas de navegación conjunta o respondan de modo dinámico a a su entorno, por ejemplo evadiendo obstáculos.

El primer enfoque al estudio del control de formaciones se basó en aproximaciones tales como el método llamado \emph{behaviour-based}. Tiene dos dificultades, la primera es que se puede volver intratable cuando aumenta el número de agentes. La segunda es que no da garantías formales de convergencia. Para un sistema multiagente, desde un punto de vista práctico, es vital poder dar garantías de que el sistema se va a comportar del modo esperado. Los contrario nos puede llevar a una multi-catástrofe.

Un gran paso adelante se dió al aplicar teoría de consenso al control de formaciones. Las aproximaciones que se han derivado de ella al control de formaciones se pueden clasificar de acuerdo a cómo se define la formación objetivo.

Tres aproximaciones clásicas son las basadas en desplazamiento, basadas en distancia y basadas en orientación. En ellas la formación objetivo se define aplicando ligaduras \emph{constantes} a los desplazamiento, a las distancias, o a las orientaciones entre los agentes.

La invarianza de estas ligaduras constantes tiene un impacto crítico en la maniobrabilidad de la formación. Por ejemplo, las restricciones al desplazamiento entre los agentes son invariantes a translaciones de la formación. Pero son difícilmente aplicables si queremos cambiar la escala o la orientación de la formación, porque requieren cambiar las ligaduras al desplazamiento entre los agentes. Las basadas en distancia entre los agentes, permiten translaciones y cambios de orientación, pero no son fácilmente aplicables si se quiere reescalar la formación.Las basadas en orientación pueden controlar translaciones y cambios de escala, pero no cambios de orientación.

Un avance claro a estas restricciones lo constituye en dos dimensiones el uso de la Laplaciana compleja (de la cual de momento pasamos) y el empleo de la matriz de tensiones (stress matrix).

La matriz de tensiones de una formación se puede ver como la matriz laplaciana generalizada de un grafo\footnote{Definimos todo más adelante, cuidadosamente.}. La estructura de la matriz de tensiones la determina el grafo subyacente a la formación, pero los valores de sus entradas los vinene determinados conjuntamente por la configuración de la formación. A diferencia de la matriz laplaciana convencional de un grafo, las entradas de la matriz de tensiones, correspondientes al peso asignado a cada eje del grafo, puede tomar valores positivos, negativos o cero.

El uso de las matrices de tensiones es muy atractivo ya que son invariantes a cualquier transformación afin de la configuración de una formación. Eso quiere decir que que son invariantes a rotaciones, translaciones, cambios de escala, cizallas y cualquier composición de ellas. Vamos a ver si logramos aclararnos con todo ello.


\section{Notación y definiciones preliminares}
\begin{enumerate}
\item Dado un conjunto cualquiera $\mathcal{X}$ llamamos $\vert \mathcal{X} \vert$ a su cardinal.
\item $\mathbf{1}_n \in \mathbb{R}^p$ vector columna de todo unos. 
\item Dado un vector, $x$ $\text{diag}(x)$ representa la matriz diagonal construida de modo que los elementos de la diagonal son los correspondientes del vector $\text{diag(x)}_{ii} =x_i$

\item $I_d \in \mathbb{R}^{d\times d}$, es la matriz identidad.

\item $\otimes$ representa el producto de Kroneker de dos matrices,
\begin{equation}
A\otimes B = \begin{pmatrix}
a_{11}B & a_{12}B &\cdots& a_{1n}B\\
\vdots  &\vdots &\ddots & \vdots\\
a_{m1}B &a_{m2}B &\cdots & a_{mn}B
\end{pmatrix}
\end{equation}
$A\in \mathbb{R}^{n\times m},\ B\in \mathbb{R}^{k\times l} \rightarrow A\otimes B \in \mathbb{R}^{nk\times ml}$
\item Consideramos un grupo de $n$ agentes móviles en $\mathbb{R}^d$ con $d\geq 2$ y $n \geq d+1$.          
\item Representamos por $p_i \in \mathbb{R}^d$ la posición del agente $i$.
\item Definimos el vector de posiciones apiladas $p =[p_1^T,\cdots,p_n^T] \in \mathbb{R}^{dn}$ como la \emph{configuración} de los agentes.
\item Definimos la interacción entre los agente empleando un grafo fijo $\mathcal{G} = (\mathcal{V},\mathcal{E})$. Donde cada agente esta asociado a un vértice del grafo $\mathcal{V}= \{1,\cdots,n\}$ y cada eje $(i,j)$ del conjunto $\mathcal{E} \subseteq \mathcal{V}\times \mathcal{V}$ indica que el agente $i$ recibe información del agente $j$. Se dice en este caso que el agente $j$ es un vecino del agente $i$.
\item El conjunto de vecinos del agente $i$ es $\mathcal{N}_i = \{j \in \mathcal{V}: (i,j) \in \mathcal{E}\}$
\item Un grafo es no dirigido si $(i,j)\in \mathcal{E} \Leftrightarrow (j,i) \in \mathcal{E}$. Mientras  no se diga lo contrario, los grafos que usamos son no dirigidos.
\item Se puede dar una orientación  un grafo dando una orientación a cada uno de los ejes no dirigidos. De esta forma, Podemos construir un conjunto ordenado (o orientado) de ejes $\mathcal{Z}$ elegiendo una de las dos direcciones arbitrarias para cada par de nodos conectados entre sí. $\mathcal{Z}_k=(\mathcal{Z}_k^{head} \mathcal{Z}_k^{tail})$, $k \in \{1,\cdots,\frac{\vert\mathcal{E}\vert}{2}\}$. El primer elemento recibe el nombre de cabeza del eje dirigido y el segundo recibe el nombre de cola.

\item Matriz de incidencia. Nosotros la construimos como un a matriz $B \in \mathbb{R}^{\vert \mathcal{V} \vert \times \vert \mathcal{Z}\vert}$ \cite{Marina2021},
\begin{equation}
b_{ik}:= \left\{ \begin{array}{ll}
+1 & \text{if } i =\mathcal{Z}_k^{tail}\\
-1 & \text{if } i = \mathcal{Z}_k^{head}\\
0  & \text{en otro caso} 
\end{array} \right.
\end{equation} 
Es interesante hacer notar que el número de filas de la matriz de incidencia coincide con el numero de agentes o de nodos del grafo, mientras que el número columnas coincide con el número de aristas orientadas. Una propiedad importante de $B$ es que $B^T\mathbf{1}_n =0$. Algunos autores como el mismísimo Zhao definen la matriz de incidencia como la traspuesta de $B$, es decir cuentas los vértices como columnas y los ejes como filas $H=B^T$. Hay que tenerlo en cuenta a la hora de seguir definiciones y demostraciones.

\item Una formación $(\mathcal{G},p)$ Es un par formado por una grafo y una configuración de modo que el vértice $i$ corresponde a la posición $p_i$ del agente $i$.

\item Definimos $\text{vec}(\cdot)$ como el vector que se obtiene al apilar todas las columnas de una matriz. Propiedades importantes: $vec(ABC) = (C^T \otimes A)\text{vec}(B)$. Con $A$,$B$ y $C$ matrices de las dimensiones adecuadas. Por tanto, $x\otimes y =\text{vec}(y,x^T)$.

\item $\text{Null}(\cdot)$ Espacio nulo de una matriz, es decir el espacio compuesto por aquellos vectores $x$ que cumplen $Ax=0, x \neq 0$.
\item $\text{Col}(\cdot)$ Espacio vectorial formado por la columnas de una matriz, tomadas como vectores.

\item $\Vert \cdot \Vert$ Norma euclidea (norma 2) de un vector y Norma expectral (norma 2) de una matriz (raíz del mayor de sus valores singulares).

\item Expansión afín. Dado un conjunto de puntos $\{p_i\}_{i=1}^n$ La expansión afín de dichos puntos se define como:
\begin{equation}
\mathcal{S} = \left\lbrace \sum_{i=1}^na_ip_i\ :\ a_i \in \mathbb{R}\ \forall i\ and \sum_{i=1}^na_i = 1 \right\rbrace
\end{equation}
(para dos puntos una recta, para tres puntos no alineados un plano, etc.)

\item Dependencia afin. Dado un conjunto de puntos $\{p_i\}_{i=1}^n$ se dice que son afinmente dependientes si Si existen escalares $\{a_i\}_{i=1}^n$ no todos nulos, tales que $\sum_{i=1}^na_ip_i =0$ y $\sum_{i=1}^na_i=0$. Si no, se dice que son afinmente independientes.
\item Matriz de configuración: A partir de un conjunto de puntos, $\{p_i\}_{i=1}^n$, se define como la matriz $P$ formada por los vectores $p_i^T$, apilados. $P\in \mathbb{R}^{n\times d}$
\begin{equation}
P(p) = \begin{bmatrix}
p_1^T\\ \vdots \\ p_n^T
\end{bmatrix}
\end{equation}

\item Matriz de configuración aumentada: $\bar{P}(p) =[P(p), \mathbf{1}_n]$. Un conjunto de puntos $\{p_i\}_{i=1}^n$ es afinmente dependiente si las filas de $\bar{P}(p)$ son linealmente dependientes, es decir, si existe $a = [a_1,\cdots, a_n]$ tales que $\bar{P}(p)a = 0$ y son afinmente independientes si las filas de $\bar{P}(p)$ lo son. Las filas de la matriz de configuración aumentada tienen $d+1$ elementos, por tanto puede haber como mucho $d+1$ puntos linealmente independientes en $\mathbb{R}^d$.

\item Si $\{p_i\}_{i=1}^n$ expande afinmente $\mathbb{R}^d$ tienen que existir en la colección $d+1$ puntos que sean afinmente independientes. En consecuencia, $\bar{P}(p)$ tiene $d+1$ filas linealmente independientes, y por tanto, tiene rango $d+1$
\begin{lem}{Condición del rango de la matriz aumentada de configuración para una exansión afín del espacio de definición.}
El conjunto de puntos $\{p_i\}_{i=1}^n$ expande afinmente el espacio $\mathbb{R}^d$ si y solo si $n \geq d+1$ y $\textrm{rank}(\bar{P}(p)) = d+1$.
\end{lem}
\end{enumerate}


\bibliographystyle{apalike}
\bibliography{bibliomultiagentes}
\end{document}